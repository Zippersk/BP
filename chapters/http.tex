\documentclass[../projekt.tex]{subfiles}
\begin{document}


\section{Hypertext Transfer Protocol (HTTP)}\label{http}
Pôvodne bol protokol HTTP určený pre výmenu dokumentov vo formáte HTML, ale v~súčasnosti sa používa aj pre prenos iných informácií. Vďaka rozšíreniu MIME (Multipurpose Internet Mail Extensions) je tento protokol schopný prenášať akýkoľvek súbor \cite{httpRFC}.

\subsection{Princíp protokolu HTTP}
Protokol HTTP funguje na princípe dotaz-odpoveď. Užívateľ (klient/user-agent) pošle serveru dotaz. Ten server spracuje a klientovi odpovie. V~odpovedi server popisuje výsledok dotazu informáciami, či sa podarilo žiadaný zdroj nájsť, či zdroj existuje, v~akom formáte je telo odpovede atď.


\subsection{Rozšírenie HTTPS}
Protokol HTTP neumožňuje zabezpečené spojenie, preto sa často používa protokol TLS (Transport Layer Security) nad vrstvou TCP. Vďaka tomu je možné vytvoriť šifrovaný kanál. Toto spojenie je označované ako HTTPS \cite{httpsRFC}.


\subsection{Stavové kódy protokolu HTTP}
Úspešnosť dotazu vieme zistiť podľa stavových kódov, ktoré sú pribalené v~odpovedi servera na dotaz klienta. Vďaka stavovým kódom vieme presne určiť, či behom spracovávania dotazu došlo k~chybe. Tým pádom môže klient reagovať na chyby.

Zoznam stavových kódov má na starosti organizácia IANA (Internet Assigned Numbers Authority). Jedná sa o~trojciferné číslo v~desiatkovej sústave. Prvé číslo určuje kategóriu odpovede a ostatné ju bližšie špecifikujú.

\subsubsection{1XX}
Kódy, začínajúci číslom 1, sú tzv. informačné. Indikujú že server dotaz spracoval a pochopil. Bližší význam záleží na zvyšných dvoch číslach. V~niektorých prípadoch môže klientovi naznačovať, že sa finálna odpoveď ešte spracováva a má na ňu počkať. Tiež môže naznačovať zmenu protokolu.

\label{2XX}
\subsubsection{2XX}
Stavové kódy začínajúce číslom 2 indikujú, že dotaz bol serverom obdržaný, pochopený a správne vyhodnotený.

\subsubsection{3XX}
Tieto kódy naznačujú, že na získanie požadovaného zdroja, je potrebné vykonať ďalšiu akciu. Zvyčajne sa jedná o~presmerovanie.

\label{4XX}
\subsubsection{4XX}
Stavové kódy, ktoré začínajú číslom 4, indikujú, že nastala chyba na strane užívateľa. Ďalšie 2 čísla presnejšie určujú o~akú chybu ide. Najčastejšie sa jedná o~chybu \texttt{404 Not Found}, ktorá hovorí že žiadaný zdroj nebol na serveri nájdený, ale môže ísť aj o~menej časté chyby ako \texttt{429 Too Many Requests}, ktorá sa vyskytuje v~prípade, že uživateľ žiadal o~daný zdroj príliš veľa krát v~určitom časovom úseku.

\label{5XX}
\subsubsection{5XX}
Stavové kódy začínajúce číslom 5, hovoria o~tom, že došlo k~chybe na strane servera. Aj keď dotaz mohol byť validný, server ho nedokázal spracovať. Mohlo sa tak stať napríklad kvôli výpadku servera (preťaženie, údržba).

\subsection{Druhy žiadostí/metódy HTTP}
Protokol HTTP využíva niekoľko žiadostí, z~ktorých najčastejšie sú:
\begin{itemize}
    \item \textbf{GET} - ide o~najbežnejší typ žiadostí. Jej výsledkom je žiadaný zdroj uvedení v~dotaze URL. Tento typ dotazu neobsahuje telo správy.
    \item \textbf{POST} - k~dotazu je pridané telo správy. Zvyčajne obsahuje hodnoty z~HTML formulára.
    \item \textbf{PUT} - využíva sa na nahranie súboru na určitú URI.
    \item \textbf{DELETE} - tento typ žiadosti je len zriedka implementovaní. Zmaže zdroj uvedení v~URI.
    \item \textbf{HEAD} - ide o~podobný typ žiadosti ako GET, ale na rozdiel od GET, server vracia len hlavičku odpovede.
    \item \textbf{OPTIONS} - v~odpovedi vracia metódy, ktoré sú povelené na danej URI.
\end{itemize}




\newpage

\subsection{Príklad komunikácie}

Klient začína komunikáciu poslaním dotazu na server. Na výpise \ref{lst:httpReqExample} sa nachádza ukážka dotazu, ktorý obsahuje zvyčajne viac informácií, ale pre zjednodušenie príkladu nie sú uvedené. Server v~dotaze špecifikuje metódu žiadosti, svoju totožnosť (Opera verzia 10.60) a podporované kódovanie.
\begin{lstlisting}[caption=príklad HTTP dotazu,frame=tlrb,label={lst:httpReqExample}]
GET / HTTP/1.1
Host: www.fit.vutbr.cz
User-Agent: Opera/9.80 (Windows NT 5.1; U; sk) Presto/2.5.29 Version/10.60
Accept-Charset: UTF-8,*
\end{lstlisting}
Príklad nasledovnej reakcie servera na dotaz sa nachádza na výpise \ref{lst:httpResExample}. Server odpovedá stavovým kódom \texttt{200 OK}, čo značí, že dotaz sa podarilo úspešne spracovať. Ďalej hlavička odpovedi okrem iného obsahuje dátum a čas vybavenia žiadosti, a informácie o~vrátenom zdroji ako typ (text/HTML), použité kódovanie (UTF-8) a dĺžku odpovede. 
\begin{lstlisting}[caption=príklad HTTP odpovedi,frame=tlrb,label={lst:httpResExample}]
HTTP/1.1 200 OK
Content-Length: 3059
Server: GWS/2.0
Date: Sat, 11 Jan 2003 02:44:04 GMT
Content-Type: text/html
Cache-control: private
Connection: keep-alive
\end{lstlisting}
\end{document}