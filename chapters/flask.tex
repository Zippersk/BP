\documentclass[../projekt.tex]{subfiles}
\begin{document}


\section{Flask}\label{flask}
Flask je volne šíriteľný mikro framework napísaný v~jazyku Python, ktorý vznikol v~roku 2004. Aj keď ešte stále nevyšla verzia 1.0, svoje uplatnenie si našiel vo veľkých spoločnostiach ako Pinterest alebo LinkedIn. Od polovice roku 2016 je najpopulárnejším Python repozitárom na GitHube.
\newline
\newline
Flask si zakladá na svojej jednoduchosti. Serverová aplikácia, ktorá po navštívení URL adresy \texttt{/zdroj} vypíše vetu "\texttt{Tu budu data zdroja.}", môže vyzerať následovne.
\begin{lstlisting}[language=Python, caption=Jednoduchá aplikácia,frame=tlrb]
from flask import Flask
app = Flask(__name__)

@app.route("/zdroj")
def zdroj():
    return "Tu budu data zdroja."
\end{lstlisting}


\subsection{Rozšírenia}
Kedže sa Flask považuje za micro framework, sám o~sebe neobsahuje pokročilé funkcionality ako abstraktnú databázovú vrstvu, validáciu formulárov, alebo stránkovanie výsledkov z~kolekcie. Avšak Flask podporuje rozšírenia, ktoré dokážu aplikáciam pridať požadované funkcionality. V~súčasnosti existuje množstvo rozšírení, ako napríklad validácia formulárov, spracovanie nahrávaných súborov alebo rozšírenia pre rôzne formy autentifikácií.


\subsubsection{Flask-RESTPlus}
Flask-RESTPlus je populárne rozšírenie frameworku Flask. Poskytuje kolekciu dekoratérov a nástrojov na opis RESTful API. Taktiež dokáže zo zdrojového kódu za behu generovať dokumentáciu vo formáte Open Api Specification 2.0.




\subsubsection{Flask-SQLAlchemy}
Jedná sa o~rozšírenie, ktoré výrazne uľahčuje prístup k~databáze, vytvorením abstraktnej vrstvy medzi databázou a aplikáciou. Zároveň výrazne neovplyvňuje výkon a flexibilitu aplikácie.


\end{document}