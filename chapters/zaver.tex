\documentclass[../projekt.tex]{subfiles}
\begin{document}


\chapter{Záver}\label{zaver}
V~tejto práci sa mi podarilo navrhnúť implementáciu serverovej časti systému Fitcrack s~automaticky generovanou dokumentáciou. Zameral som sa na vylepšenia súčasnej verzie webového prostredia Fitcracku a navrhol som rozšírenia, ku príkladu systém viacerých užívateľov, ktoré spríjemnia užívateľom používanie. Taktiež som odstránil nedokonalosti súčasného riešenia, ako napríklad dostupnosť len z~webovej platformy. Systém, implementovaný podľa môjho návrhu bude dostupný zo všetkých platforiem, ktoré sú schopné sieťovej komunikácie. Súčasne sa zníži záťaž na server, pretože bude prerozdelená medzi klienta a server. Vďaka implementácie pomocou modulov bude celý systém ľahko rozšíriteľný. Návrh opísaný v~tejto práci tiež ošetruje niekoľko bezpečnostných dier súčasnej implementácie. Návrh spĺňa všetky zásady architektúry REST (viď \ref{zasadyREST}), a taktiež je vhodný na testovacie účely systému.

Keďže som v~rámci tejto práce zhromaždil potrebné znalosti, v~nasledujúcom semestri budem pracovať na implementácií navrhnutého systému. Tiež sa budem venovať testovaniu spoľahlivosti a experimentom. Systém podrobím výkonnostným testom a výsledky porovnám so súčasným riešením. 
Mojou snahou je vytvoriť aplikačné rozhranie na vzdialené ovládanie systému Fitcrack takým spôsobom, aby implementácia klientskej časti užívateľského rozhrania bola jednoduchá a vyžadovala čo najmenej informácií o~štruktúre serverovej časti. 

\end{document}